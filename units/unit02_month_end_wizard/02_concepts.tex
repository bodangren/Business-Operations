% Update: Draft full prose Concepts section for Unit 2
\section{Concepts}
\label{sec:unit2_concepts}

In Unit 2, you’ll shift from manual bookkeeping to designing an automated month-end close tool that compresses a multi-day process into a single click. This Concepts section lays the groundwork for understanding \emph{why} each accounting adjustment exists, how depreciation and closing entries tie the financial statements together, and why Excel’s structured tables, named ranges, and macros are the perfect allies in this quest. Though automation may sound purely technical, it rests on solid accounting principles—so we’ll explore both the theory and the practical application in tandem.

\subsection{The Rationale for Accrual Accounting}
Accrual accounting ensures that revenues and expenses are recognized in the periods to which they truly belong, not simply when cash moves through your bank account. This matching principle is critical for producing financial statements that accurately reflect a company’s performance and position.

When you speak with your CFO at the Entry Event, they’ll emphasize that failing to record an expense in the period it was incurred can misstate profitability and mislead stakeholders. Likewise, recognizing revenue too early can inflate earnings. In practice, we encounter four types of adjustments:
\begin{itemize}
  \item \textbf{Accrued Expenses}, which record costs incurred but not yet paid (e.g., utilities used in March but billed in April).
  \item \textbf{Prepaid Expenses}, where cash is paid upfront but the benefit spans future periods (e.g., insurance premiums).
  \item \textbf{Accrued Revenues}, for services rendered or goods delivered before billing (e.g., consulting hours provided at month-end).
  \item \textbf{Unearned Revenues}, where cash is received before delivering the product or service (e.g., subscription fees).
\end{itemize}

\begin{Definition}
\textbf{Accrual:} The recognition of revenue or expense before cash exchange.  
\textbf{Deferral:} The postponement of revenue or expense recognition until a future period.
\end{Definition}

Automating these entries means your macro or recorder will systematically apply formulas or VBA routines that calculate exactly how much to adjust, based on criteria you’ve defined in named ranges and tables.

\subsection{Calculating Depreciation with SLN}
Depreciation spreads the cost of a long-lived asset over its useful life. While there are many methods (double-declining balance, units of production), Unit 2 focuses on the simplest: straight-line depreciation. The straight-line method divides the depreciable base (cost minus salvage value) equally across periods.

\[
\text{Periodic Depreciation} = \frac{\text{Cost} - \text{Salvage}}{\text{Useful Life}}
\]

In Excel, the built-in function \verb|SLN(cost, salvage, life)| returns the exact depreciation per period. Embedding this function in your model means that when you change the cost, salvage value, or useful life—perhaps through a UI control—your depreciation schedule will update automatically.

\begin{Example}
A company buys equipment for \$10,000 with a \$2,000 salvage value and a five-year useful life.  
\verb|=SLN(10000, 2000, 5)| → \$1,600 depreciation each year.
\end{Example}

By automating SLN via macros or named‐range–driven formulas, you ensure consistency and eliminate manual copying errors.

\subsection{Purpose and Structure of Closing Entries}
At the end of each period, temporary accounts—revenues, expenses, and dividends—must be closed to retained earnings to start the new period at zero. Closing entries accomplish this reset:

\begin{enumerate}
  \item Transfer all revenue balances to Income Summary (debit revenues, credit Income Summary).
  \item Transfer all expense balances to Income Summary (credit expenses, debit Income Summary).
  \item Transfer the net Income Summary balance to Retained Earnings.
\end{enumerate}

\begin{Warning}
If you skip or misapply closing entries, temporary account balances will carry forward, distorting next period’s results and violating GAAP.
\end{Warning}

Designing a macro to insert these entries requires understanding each step’s logic—what accounts to target, how to calculate the total, and where to place the generated journal lines.

\subsection{Excel Tables and Named Ranges as Foundations for Automation}
Excel Tables transform flat data ranges into dynamic objects. When you convert your raw receipts or preliminary ledger entries into a Table:
\begin{itemize}
  \item New rows automatically adopt formatting and formulas.  
  \item Columns become first-class objects you can reference by name.  
  \item Filters, sorting, and structured references simplify both manual review and macro code.
\end{itemize}

Naming key ranges—such as \texttt{AdjEntries} for your adjusting entries table or \texttt{DepSchedule} for depreciation output—not only makes formulas readable but also allows macros to refer to data logically (e.g., \verb|Range("DepSchedule")|) instead of hard-coded addresses.

\begin{Tip}
After creating your Table with \texttt{Ctrl+T}, click the Table’s header and assign it a meaningful name (e.g., \texttt{ReceiptsTbl}) in the Name Box.
\end{Tip}

This practice establishes a sturdy framework upon which your automation routines will run reliably, even if you insert or delete rows.

\subsection{Leveraging the Macro Recorder vs. Direct VBA}
For Excel novices, the Macro Recorder is a gateway to automation: it captures every action—cell selection, formatting, formula entry—as VBA code. However, recorded macros often include extraneous steps (selecting cells, copying and pasting) and lack control structures.

\begin{Important}
Use the Recorder to scaffold your process, then refine the generated code by removing unnecessary selections, adding loops to process entire tables, and inserting \verb|If...Then| statements for error trapping.
\end{Important}

Direct VBA editing unlocks deeper capabilities: you can write functions to iterate through each named range, check for missing descriptions, or even launch a custom form for user inputs.

\subsection{Creating a User Interface with Buttons and Form Controls}
A polished Month-End Wizard feels like an application, not a spreadsheet. By adding a Form Control button linked to your \texttt{CloseMonth} macro, you allow any user—accounting intern or CFO—to trigger the entire closing sequence without navigating the Developer tab.

\begin{Example}
Insert → Form Controls → Button. In the assignment dialog, choose \texttt{CloseMonth}. Right-click to edit the button text to “Run Month-End Wizard.”
\end{Example}

Combining button placement with data validation dropdowns or input boxes enables users to specify parameters (e.g., the period to close) before running the automation.

\subsection{Integrated Workflow: From Raw Data to Automated Close}
When you bring all these concepts together, your workflow follows a logical chain:

\begin{enumerate}
  \item \textbf{Import Raw Receipts:} Load CSV into \texttt{ReceiptsTbl}.  
  \item \textbf{Map Adjusting Entries:} Use named ranges and the macro to calculate accruals, deferrals, and depreciation.  
  \item \textbf{Execute SLN Schedules:} Populate \texttt{DepSchedule} via formulas recorded or coded in VBA.  
  \item \textbf{Insert Closing Entries:} Run the \texttt{CloseMonth} macro to zero out temporary accounts.  
  \item \textbf{Trigger with UI Button:} Click “Run Month-End Wizard” to execute the full sequence in under two minutes.
\end{enumerate}

Each step builds on the previous one, ensuring that by the end of Week 4 you’ll not only have a functioning Month-End Wizard but also a deep understanding of the accounting mechanics and Excel architecture that make such automation possible.

\clearpage
