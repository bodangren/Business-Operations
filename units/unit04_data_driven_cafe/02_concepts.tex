% Update: Unit 4 Concepts section for Data-Driven Café

\section{Concepts}
\label{sec:unit4_concepts}

In Unit 4, you’ll learn how to turn raw point-of-sale (POS) data into actionable insights that drive operational decisions. This Concepts section lays out the statistical foundations, data-structuring principles, and forecasting techniques that underpin your café inventory and staffing plan. Think of these ideas as the engine behind your models: without a solid grasp of how data behaves and how forecasts are generated, any recommendation you make could miss the mark.

\subsection{Purpose and Scope of Data-Driven Decision-Making}
Modern foodservice operations must balance profitability with waste reduction. A data-driven approach uses historical sales records to anticipate demand, optimize staffing levels, and minimize spoilage. By analyzing two years of weekend POS data, you gain both the granularity and the context needed to make recommendations that maximize profit while capping waste at 3 %.

\begin{Trivia}
According to a 2021 study, restaurants that adopted sales forecasting reduced food waste by an average of 23 % within six months.
\end{Trivia}

Your goal is not merely to produce numbers, but to craft a believable, evidence-backed story: “Here’s what happened, here’s what’s likely to happen, and here’s what we should do about it.”

\subsection{Data Cleaning and Validation}
Raw CSV exports often contain inconsistencies: missing values, stray characters, duplicate records, or misaligned columns. Before any analysis, data must be cleaned:

\begin{itemize}
  \item \textbf{Trim and normalize text:} Remove leading/trailing spaces, standardize date formats.
  \item \textbf{Detect duplicates:} Ensure each transaction appears only once.
  \item \textbf{Handle missing values:} Decide whether to impute, ignore, or flag them.
\end{itemize}

\begin{Definition}
\textbf{Data validation:} The process of checking for and correcting errors in a dataset before analysis.
\end{Definition}

\begin{Example}
If two identical transaction IDs appear with the same timestamp, one is likely a duplicate—dropping it prevents skewing total sales.
\end{Example}

\begin{Tip}
Use \verb|Text-to-Columns| and \verb|TRIM()| in Excel to standardize dates and remove extraneous spaces in one step.
\end{Tip}

\subsection{Descriptive Statistics Fundamentals}
Once your dataset is clean, descriptive statistics summarize its key characteristics:

\begin{itemize}
  \item \textbf{Mean:} The average sales per weekend.
  \item \textbf{Median:} The midpoint of your sales distribution.
  \item \textbf{Standard deviation (σ):} How much weekend sales typically vary.
\end{itemize}

\begin{Definition}
\textbf{Standard deviation:} A measure of dispersion, indicating the typical distance of data points from the mean.
\end{Definition}

\begin{Example}
If the average weekend revenue is \$1,200 with σ = \$200, about 68 % of weekends will fall between \$1,000 and \$1,400.
\end{Example}

\subsection{Outlier Detection with Z-Scores}
Not every datapoint deserves equal weight. Extremely high or low values can distort your model. A z-score measures how many standard deviations a value lies from the mean:

\[
z = \frac{x - \mu}{\sigma}
\]

\begin{Definition}
\textbf{Z-score:} The number of standard deviations a data point \(x\) is from the mean \(\mu\).
\end{Definition}

\begin{Example}
A weekend with \$1,800 in sales when \(\mu=1200\) and \(\sigma=200\) yields \(z=(1800-1200)/200=3\), marking it as an outlier.
\end{Example}

\begin{Warning}
A high z-score doesn’t always indicate an error—it may reflect a genuine surge in demand (e.g., a holiday weekend).
\end{Warning}

\subsection{Data Visualization: Histograms and Box-Plots}
Visual tools reveal patterns that raw numbers cannot:

\begin{itemize}
  \item \textbf{Histogram:} Shows the frequency distribution of weekend sales.
  \item \textbf{Box-plot:} Highlights median, quartiles, and potential outliers.
\end{itemize}

\begin{Example}
A histogram revealing a bimodal distribution indicates two common sales levels—perhaps reflecting weekday-vs-weekend menu differences.
\end{Example}

\begin{Tip}
Use the Analysis ToolPak’s histogram tool to automate bin selection, then overlay a box-plot to compare variability over two years.
\end{Tip}

\subsection{Regression Forecasting Fundamentals}
Linear regression fits a line through your data to predict future values:

\[
\hat{y} = \beta_0 + \beta_1 x
\]

where \(x\) is time (weekend number) and \(\hat{y}\) is projected sales.

\begin{Definition}
\textbf{Regression:} A statistical technique to model the relationship between a dependent variable and one or more independent variables.
\end{Definition}

\begin{Example}
Fitting a regression line to weekend-by-weekend sales may reveal a slight upward trend reflecting growing customer loyalty.
\end{Example}

\begin{Important}
Regression assumes a linear relationship—if your data shows seasonality, you may need additional techniques (e.g., moving averages).
\end{Important}

\subsection{Scenario Simulation and Cost Analysis}
Once you have a forecast, simulate how different staffing or ordering decisions affect profit and waste:

\begin{itemize}
  \item \textbf{Staffing simulation:} Project labor cost under low, average, and high demand scenarios.
  \item \textbf{Inventory simulation:} Calculate holding cost and spoilage if you overstock by a given percentage.
\end{itemize}

\begin{Example}
Simulating a 10 \% overstock in pastries might show a \$150 variance in waste cost versus a 5 \% understock showing \$200 in lost sales.
\end{Example}

\subsection{Integrating Insights into Recommendations}
The final step is narrative: translate numbers into clear actions:

\begin{enumerate}
  \item Summarize key statistical findings (mean, sd, outliers).
  \item Explain forecast implications (upward/downward trends).
  \item Recommend concrete staffing hours and order quantities.
  \item Justify each recommendation with cost-benefit estimates.
\end{enumerate}

\begin{Tip}
Frame your report as “We forecast X, which suggests ordering Y pastries and scheduling Z staff hours to optimize profit and minimize waste.”
\end{Tip}

\begin{Important}
Your audience (café manager) cares less about formulas than about confidence: show how your simulation narrows uncertainty to under ±5 %.
\end{Important}

%TODO: Link to POS dataset, z-score guide, Analysis ToolPak tutorial, histogram & box-plot templates, regression forecast guide, cost-simulation workbook, and infographic pitch template.

\clearpage
