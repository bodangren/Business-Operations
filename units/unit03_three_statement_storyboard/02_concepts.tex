% Update: Concepts section for Unit 3 — Three-Statement Storyboard

\section{Concepts}
\label{sec:unit3_concepts}

In Unit 3 you will learn how individual journal entries connect across the core financial statements to tell a cohesive story of profitability, financial position, and cash health. This Concepts section lays out the theoretical foundations and vocabulary you need before building dynamic links and dashboards in Excel. Think of this as understanding the “plot” behind each number: once you know how data flows, you can craft models that truly persuade investors.

\subsection{Financial Narrative \& Statement Flow}
Every transaction your business records pushes through three statements in sequence:

\begin{enumerate}
  \item \textbf{Income Statement:} Measures performance over a period (revenues less expenses).
  \item \textbf{Balance Sheet:} Captures position at a point in time (assets, liabilities, equity).
  \item \textbf{Cash Flow Statement:} Reconciles net income to cash movements.
\end{enumerate}

These are not isolated reports but a continuous loop: net income from the Income Statement increases (or decreases) Retained Earnings on the Balance Sheet, and the Indirect Cash Flow Statement explains how accrual-based income translates to actual cash.

\begin{Definition}
\textbf{Financial Narrative:} The story that links profitability (Income Statement) to solvency (Balance Sheet) and liquidity (Cash Flow Statement), demonstrating how value is created and sustained.
\end{Definition}

\begin{Example}
A \$1,000 sale recorded on credit raises Revenue on the Income Statement. Retained Earnings grows by \$1,000 on the Balance Sheet, but no cash appears until the receivable is collected—this timing gap is highlighted on the Cash Flow Statement.
\end{Example}

\subsection{Income Statement Concepts}
The Income Statement summarizes all revenues earned and expenses incurred:

\begin{itemize}
  \item \emph{Revenue accounts} capture inflows earned but not necessarily received in cash.
  \item \emph{Expense accounts} match costs to the period in which they help generate revenue (matching principle).
  \item \emph{Gross profit}, \emph{operating income}, and \emph{net income} illustrate progressive subtotals.
\end{itemize}

\begin{Tip}
Use structured references (e.g., \verb|Income[Amount]|) in Excel Tables to ensure formulas auto-expand as you add new line items.
\end{Tip}

\begin{Warning}
Omitting an expense category (e.g., depreciation) understates costs and overstates profit—always verify account completeness.
\end{Warning}

\subsection{Balance Sheet Linking \& Retained Earnings}
The Balance Sheet balances because:
\[
  \text{Assets} = \text{Liabilities} + \text{Equity}
\]
Within Equity, the Retained Earnings line rolls forward net income from prior periods:

\[
  \text{Ending RE} = \text{Beginning RE} + \text{Net Income} - \text{Dividends}
\]

Linking Net Income to Retained Earnings is critical for model integrity.

\begin{Definition}
\textbf{Retained Earnings:} Cumulative profits reinvested in the business rather than distributed.
\end{Definition}

\begin{Example}
If Beginning RE is \$5,000 and Net Income this period is \$1,200, then Ending RE becomes \$6,200 (assuming no dividends).
\end{Example}

\subsection{Indirect Cash Flow Statement}
The Indirect Method adjusts net income to cash basis:

\begin{enumerate}
  \item Start with Net Income.
  \item Add back non-cash expenses (depreciation, amortization).
  \item Subtract increases in working capital (e.g., accounts receivable).
  \item Add decreases in working capital (e.g., inventory sold).
\end{enumerate}

This shows how accrual profits convert into cash.

\begin{Definition}
\textbf{Indirect Cash Flow:} A statement that reconciles accrual net income to net cash from operating activities.
\end{Definition}

\begin{Example}
Depreciation of \$300 increases cash flow (added back) even though it reduced net income.
\end{Example}

\begin{Tip}
Set up each adjustment as a separate line in your Excel Table so that readers can trace every change from accrual to cash.
\end{Tip}

\subsection{Cross-Sheet References \& INDEX/MATCH}
To connect statements:

\begin{itemize}
  \item Use \verb|INDEX/MATCH| instead of hard-coded cell links for flexibility.
  \item Name key ranges (e.g., \verb|Balance!$B$2:$B$20| as \verb|Assets|).
  \item Reference Income Statement totals directly in the Balance Sheet using names, not cell addresses.
\end{itemize}

\begin{Tip}
Create a centralized \verb|Assumptions| sheet to store constants (e.g., tax rate) and reference them by name—for easier scenario updates.
\end{Tip}

\subsection{Key Performance Indicators (KPIs) \& Dashboards}
A dashboard translates raw statements into actionable metrics:

\begin{itemize}
  \item \emph{Current Ratio} \(=\frac{\text{Current Assets}}{\text{Current Liabilities}}\) gauges short-term solvency.
  \item \emph{Return on Assets} \(=\frac{\text{Net Income}}{\text{Average Assets}}\) measures efficiency.
  \item \emph{Cash Runway} \(=\frac{\text{Cash Balance}}{\text{Average Monthly Burn}}\) estimates survival time.
\end{itemize}

Visualize these with sparklines, data bars, and charts on a single dashboard sheet.

\begin{Example}
A Current Ratio below 1.0 appears as a red data bar, signaling potential liquidity issues.
\end{Example}

\subsection{Integrated Workflow Recap}
Your end-to-end process will be:

\begin{enumerate}
  \item \textbf{Build Income Statement} using journal entries and structured SUM formulas.
  \item \textbf{Link to Balance Sheet} and reconcile Retained Earnings via cross-sheet INDEX/MATCH.
  \item \textbf{Prepare Cash Flow} by laying out accrual-to-cash adjustments.
  \item \textbf{Design KPI Dashboard} with named ranges, conditional formatting, and dynamic charts.
  \item \textbf{Validate Model Integrity} through trace precedents and trial runs of scenarios.
\end{enumerate}

\begin{Important}
Understanding the flow between statements is your strongest guard against model errors. Each link you build adds credibility to your financial narrative.
\end{Important}

%TODO: Link to Excel templates for Income Statement, Balance Sheet, Cash Flow; KPI Dashboard starter; INDEX/MATCH primer; peer critique checklist.

\clearpage
