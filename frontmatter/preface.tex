% frontmatter/preface.tex
\cleardoublepage
\chapter*{Preface}

Welcome to \emph{Math for Business Operations: Applied Accounting with Excel}.  This book is designed to guide you through real-world financial challenges using Excel as your modeling tool.  Each two-week unit poses an authentic question (the Driving Question), shows you how to set up and automate your spreadsheet model, walks through worked examples, and gives you exercises to practice and reflect on your learning.

\medskip
Each unit follows this structure:
\begin{itemize}
  \item \textbf{Introduction \& Entry Event:} A real business scenario or challenge to spark your curiosity.
  \item \textbf{Concepts \& Skills:} Clear explanations of accounting or data-analysis ideas alongside hands-on Excel techniques.
  \item \textbf{Step-by-Step Examples:} Detailed walkthroughs showing you exactly how to build and test your model.
  \item \textbf{Practice Exercises:} Problems at varied difficulty levels to reinforce and extend your understanding.
  \item \textbf{Summary \& Reflection:} Key takeaways, reflection prompts, and suggestions for next steps.
\end{itemize}

Throughout the text you will also find:
\begin{itemize}
  \item \emph{Tip} boxes with shortcuts and best practices.
  \item \emph{Important} callouts highlighting common pitfalls.
  \item Excel starter files, cheat-sheets, and templates to download and explore.
  \item A glossary of key terms and an index to help you look up concepts quickly.
\end{itemize}

Use this book alongside the provided Excel workbooks: open the starter file, follow the examples step by step, then tackle the exercises on your own.  Review the reflection questions before moving on, and don’t hesitate to revisit earlier sections if you need to strengthen your skills.  By the end of the year, you will have built investor-ready models, mastered automated checks, and developed the confidence to analyze real financial data.

\addcontentsline{toc}{chapter}{Preface}
\clearpage
